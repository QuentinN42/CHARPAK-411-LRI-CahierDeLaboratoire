\subsubsection*{Le matin :}
Ce matin, Roman \textsc{Bresson}, un thesard m'a présenté sa thèse : \\


Données utilisées:
\begin{itemize}
    \item Data set public
    \item Entre 6 et 15 critères.
    \item Entre 200 et 1600 donnés
\end{itemize}


Des informations m'ont étés comuniquées :
\begin{itemize}
    \item Choquet de base en python
    \item Ctypes pour importer des fonctions C dans python
    \item Un article pour comprendre les integrales de choquet\cite{grabisch2016fuzzy}
\end{itemize}


Le réseau de neurones utilisé est configuré de la maniere suivante :
\begin{itemize}
     \item Vecteur X en entré
     \item Réseau de neurone -> dit si une le choix est "bon" ou pas.
     \item Extraction du réseau : les `ui` et les `wij` de par son architecture.
     \item Regression : testé avec des gaussiene et des sigmoides
     \item Modele : trois sommes :
        \begin{equation}
            \label{choquet}
            C_n (X)  =
            \sum_{i=1}^{n}
                w_i \times x_i +
            \sum_{i=1}^{n}\sum_{j=i+1}^{n}
            \Big(
                w_{M\,ij} \times \max(x_i,x_j) + w_{m\,ij} \times \min(x_i,x_j)
            \Big)
        \end{equation}
    \item Probleme : \\
        On ne sait pas combien en mettre de regression pour etre au plus proche des donné sans faire de l'overfeeting.
\end{itemize}

\subsubsection*{L'après midi :}
Un premier réseau de neurones à été créé afin que je me familliarise avec ces nouvelles notions.
J'utilise la librairie keras\cite{keras} (Python).


Le premier reseau est assez simple : Il fait une somme pondérée.
C'est la première partie de la formule\ref{choquet}, j'ai aussi trouvé un lien pour faire le max/min de plusieurs
noeuds mais je ne sait pas l'implementer.
