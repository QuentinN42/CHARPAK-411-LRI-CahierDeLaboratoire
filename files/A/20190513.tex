Ce matin, Roman \textsc{Bresson}, un thesard m'a présenté sa thèse : \\


Données utilisées:
\begin{itemize}
    \item Data set public
    \item Entre 6 et 15 critères.
    \item Entre 200 et 1600 donnés
\end{itemize}


Des informations m'ont étés comuniquées sur le python :
\begin{itemize}
    \item Choquet de base
    \item Ctypes pour importer des fonctions en C
\end{itemize}


Le réseau de neurones utilisé est configuré de la maniere suivante :
\begin{itemize}
     \item Vecteur X en entré
     \item Réseau de neurone -> dit si une le choix est "bon" ou pas.
     \item Extraction du réseau : les `ui` et les `wij` de par son architecture.
     \item Regression : testé avec des gaussiene et des sigmoides
     \item Modele : trois sommes :
        \begin{equation}
            \sum_{i=1}^{n}
                w_i \times u_i +
            \sum_{i=1}^{n}\sum_{j=i+1}^{n}
            \Big(
                w_{M\,ij} \times max(ui,uj) + w_{m\,ij} \times min(ui,uj)
            \Big)
        \end{equation}
    \item Probleme : \\
        On ne sait pas combien en mettre de regression pour etre au plus proche des donné sans faire de l'overfeeting.
\end{itemize}
